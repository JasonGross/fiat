\section{Computation}

The synthesis process in WhetStone is one of \textit{refinement}:
users start out with a high-level specification which can be
implemented in any number of different ways and iteratively reduces
the possible implementations until they have produced a (hopefully
efficient) fully computational implementation.

The \textsf{Comp} data type shown in
Figure~\ref{fig:Comp+Defs}\footnote{This section's core notions of
  computational refinement are found in \textsf{Computation/Core.v}.}
defines the set of computations with three constructors while
Figure~\ref{fig:ComputesTo+Defs} gives the semantics of each
constructor defined by the values it can can produce.

\begin{itemize}
\item \textsf{Return} computes to a singular value \textsf{a}, and is
  thus fully deterministic.
\item\textsf{Pick}, in contrast, represents a nondeterministic
  computation which can compute to \textit{any} value \textsf{a}
  satisfying some predicate \textsf{P} (defined as an
  \textsf{Ensemble}).
\item \textsf{Bind} builds composite computations, computing to any
  value \textsf{b} produced by binding any value \textsf{a} yielded by
  the first computation $\mathsf{c_a}$ in the second computation
  $\mathsf{c_b}$.
\end{itemize}



\begin{figure}[h]
  \begin{tabbedcodeblock}
    Inductive Comp : Type $\rightarrow$ Type := \+\\
    $|$ Return : forall A, A $\rightarrow$ Comp A \\
    $|$ Pick : forall A, Ensemble A $\rightarrow$ Comp A \\
    $|$ Bind : forall A B, Comp A $\rightarrow$ (A $\rightarrow$ Comp B) $\rightarrow$ Comp B.
  \end{tabbedcodeblock}
  \caption{The definition of computations. }
  \label{fig:Comp+Defs}
\end{figure}

\begin{figure}[h]
  \begin{tabular}{cc}
    \begin{minipage}{.5\columnwidth}
      \infrule{}{\mathsf{Comp a} \leadsto \mathsf{a}}
    \end{minipage}
    &
    \begin{minipage}{.5\columnwidth}
      \infrule{\mathsf{P~a}}{\mathsf{Pick P} \leadsto \mathsf{a}}
    \end{minipage}
  \end{tabular}
  \infrule{\mathsf{c_a} \leadsto \mathsf{a} \andalso
    \mathsf{c_b~a} \leadsto \mathsf{b}}
  {\mathsf{Bind~c_a~c_b} \leadsto \mathsf{b}}
  \caption{The semantics of computations. }
  \label{fig:ComputesTo+Defs}
\end{figure}

We can define refinement as a partial order on computations: a
computation $\mathsf{c_a}$ is a refinement of a computation
$\mathsf{c_b}$ iff $\mathsf{c_a}$ computes to a subset of the values
that $\mathsf{c_b}$ computes to:
\begin{align*}
  \mathsf{c_a \subset c_b} & \quad\quad\quad \longleftrightarrow  &
  \mathsf{\forall a, c_a \leadsto a \rightarrow
    c_b \leadsto a}
\end{align*}
\textsf{Comp} obeys the monad laws using both $\leadsto$ and $\subset$
as the notion of equivalence.

The essence of the refinement process is to refine computations by
incrementally replacing the nondeterministic parts with more
deterministic ones. We accomplish this using setoid rewriting, which
extends Coq's rewriting machinery with support relations other than
Leibniz equality\footnote{The
  \href{http://coq.inria.fr/distrib/current/refman/Reference-Manual029.html}{\underline{Coq
      documentation}} has a full explanation of the machinery
  involved.}. In particular, we can use facts about the refinement
relation $\leadsto$ to refine computations. Given a hypothesis
$\mathsf{c_a} \subset \mathsf{c_a'}$, for example, we can use the
\textsf{setoid\_rewrite} tactic to transform the goal
$\mathsf{Bind~c_a~c_b}$ into
$\mathsf{Bind~c_a'~c_b}$. \textsf{Computation/SetoidMorphisms.v}
registers $\leadsto$ as a \textsf{Parametric Relation} and also proves
that a number of different computations are \textsf{Parametric
  Morphism}s which respect this relation, allowing us to refine them
using setoid rewriting.  \textsf{ComputationExamples.v} includes an
example derivation which uses this machinery to refine a computation
which folds a binary operation, \textsf{bin\_op}, over a list.

\subsubsection{Finding the minumum element in a list}
